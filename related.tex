\section{Related Work}
\label{sec_related_work}
Our work is closely related to document classification, which is one of the oldest tasks in natural language processing. The most classic method is to combine bag-of-words features with varies classifiers \cite{joachims1998text}. Recently, neural network models like Convolutional Neural Network (CNN) \cite{kim2014convolutional} have been applied to document classification and achieve good performance. \cite{tang2015document} proposes a two-layer scheme, where they use recurrent neural network (RNN) or CNN for sentence embedding, and another RNN for document embedding. \cite{yang2016hierarchical} further adds attention mechanism to the two-layer scheme to distinguish important words or sentences from unimportant ones. As for multi-label document classification, two loss functions are commonly used. The first one is binary cross entropy \cite{nam2014large}, which treats the multi-label classification task as multiple binary classification tasks. The second one is cross entropy \cite{kurata2016improved}. In training phase, it converts the multi-label target to label distribution. After that, it uses the validation set to select a threshold, and consider all the classes with scores higher than the threshold to be positive. In our experiment, we find the latter one converges faster and performs better, so we will use the latter one in this paper.


%Our model is similar to the HAN model in that we both use hierarchical attention. However, the output of our model include multiple labels, and our model also use extra information (related law articles) to help the classification. \todo{add multi-label classification?}

Since the charge of the defendant is often part of the judgement result of a case, our task is also closely related to the thread of work on predicting the outcome of a case. Some work focuses on predicting wether the plaintiff will win or not \cite{aletras2016predicting}, while others try to predict whether the present court will affirm or reverse the decision of a lower court \cite{katz2016general}. The method can be domain specific logical model \cite{bruninghaus2003predicting}, multi-layer perceptron \cite{bench1993neural}, SVM \cite{aletras2016predicting} and random forest \cite{katz2016general}. Instead of predicting overall binary result, our task focuses on the detailed result of the case and our output may contain multiple charges.

Another related thread of work is predicting relevant law articles based on case facts. This task aims to find relevant law articles that can be applied to the given case. \cite{liu2015predicting} proposes to first use the SVM model for basic article classification, and then use some reranking methods to get the final relevant article list. In our model, however, since we use attention mechanism on the extracted articles, we only care the recall@k rather than the ranking quality in the relevant article extraction step. Therefore, we will simply use SVM for relevant article extraction in our experiment.

Our work also shares the same spirit with the legal question answering task that relevant law articles are import for decisions in civil law system. This task is proposed by the Competition on Legal Information Extraction/Entailment (COLIEE) in 2014\footnote{\url{http://webdocs.cs.ualberta.ca/~miyoung2/jurisin_task/index.html}}, and it aims at answer the yes/not questions in the in Japanese legal bar exams. The task first requires participants to extract relevant Japanese Civil Code articles, and then participants will use these articles to answer the question. The article extraction phase is often treated as an information retrieval task, and methods like tf-idf, LDA, and rank-SVM have been applied \cite{kim2014legal}. The question answering phase is often treated as a textual entailment task and methods like CNN \cite{kimconvolutional} have been used. There is another thread of work that tries to answer the multiple-choice questions in the USA National Bar Exam \cite{FAWEI16,adebayoneural}. Since the United States operates on common law system, relevant article extraction phase is not employed in these works.