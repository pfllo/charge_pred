\section{Introduction}
Determining appropriate charges (like \emph{fraud}, \emph{theft}, and \emph{homicide}) of a case is helpful for legal assistance systems when the user would like to query the system by describing the case, and it is even more important when the user has no idea of the legal background of a case, where the only input he (or she) can give is the fact description of the case. 
% In this situation, predicting appropiate charges would help us to provide the user with relevant  
For example, if the user wants to find similar cases, one can use the predicted charges of the query case to filter out irrelevant results. And if the user wants to know the possible penalties \orange{regarding a case}, one also need to decide appropriate charges first.


% For example, if the user interface is dialogue system, one can first determine suitable charges of the current case based on the user description, and then use this to guide the subsequent conversation. If the user wants to find similar cases, one can use the predicted charges of the query case to filter out irrelevant results. Furthermore, if one wants to build an end-to-end judgement suggestion system, where the input is the description of a case, he (or she) also needs to identify appropriate charges before suggesting corresponding penaties.
% \todo{add end-to-end judgement suggestion system?} \todo{find examples}

All these situations require us to predict appropriate charges of a case based solely on the fact descriptions. 
However, this fact-based charge prediction task is not easy:
(1) Multiple crimes may be involved in a single case, which means we need to conduct charge prediction in the multi-label classification paradigm. 
(2) The differences between two charges can sometimes be subtle. For example, in the context criminal cases in China, distinguishing \emph{intentional homicide} from \emph{negligent homicide} involves detailed analysis of the behavior of the defendant, and \emph{acceptance of bribes} differs from \emph{acceptance of bribes by a non-state functionary} in the occupation of the defendant. 
(3)  Although we can expect the model to implicitly learn the legal background of the judgement through massive training data, the charge prediction is still not convincing enough if no law articles are involved in the prediction. This problem is prominent in countries using civil law system, e.g., China (except Hong Kong), where the judgement is made only based on statutory law. Even in countries using common law system, e.g., the United States (except Louisiana), where the judgement is based mainly on decisions of previous cases, there are still some statutory laws that need to be followed when making judgements. 

Therefore, to solve this fact-based charge prediction task, we need a multi-label classification model, that can effectively capture the overall \orange{framework} along with important details of the fact description, and is able to extract and utilize relevant law articles to build the bridge from the fact description to appropriate charges as well. 

Our fact-based charge prediction task is closely related to the thread of work on predicting the results of a case since the judgement of a case often involves deciding appropriate charges. Previous works on this thread mainly consider a binary classification paradigm. The target is either to decide whether the outcome will side with the plaintiff or defendant \cite{aletras2016predicting}, or will the present court affirm or reverse the decision of a lower court \cite{katz2016general} \footnote{\cite{katz2016general} also use an additional \emph{other} class to represent other complex outcomes.}. Instead of case-level prediction, some researches also focus on predicting the votes of each justice \cite{martin2002dynamic,lauderdale2014scaling,sim2015utility}. Despite their binary prediction nature, these methods either do not use fact descriptions or just capture shallow semantic meaning of the facts, e.g., using Bag-of-Words (BOW). Furthermore, none of these works employ relevant law articles during prediction. Therefore, these methods are not suitable for our task. 

% To make conprehensive understanding of the fact description, \orange{we propose to use the framework of the Hierarchical Attention Network (HAN)} \cite{yang2016hierarchical} for document embedding. Specifically, we use a sentence-level and a document-level Gated Recurrent Unit (GRU) to embed each word and each sentence along with their contexts. 
To make conprehensive understanding of the fact description, inspired by previous works on document classification \cite{tang2015document,yang2016hierarchical}, we use a sentence-level and a document-level Gated Recurrent Unit (GRU) to embed each word and each sentence along with their contexts. 
Then we use attention mechanism to select the most informative words or sentences for sentence and document embedding respectively. 
To handle the multi-label nature of the problem, we convert the multi-label target to label distribution, and then use cross entropy as loss function. 
% We find this method works well in our experiments and significantly outperforms the baseline BOW method.
To get support from law articles, we first use a simple BOW-based article classifier to quickly \orange{filter out most of the irrelevant articles}. Then we attentively aggregate the retained top $k$ articles for further charge classification.
% Although the top $k$ articles are noisy, the experimental results show that our attentive aggregation module can further attend to relevant ones and thus improve the prediction performance. 

We evaluate our method by predicting the charges of criminal cases in China, \orange{which began to} officially publish the judgement documents on China Judgements Online\footnote{http://wenshu.court.gov.cn/} since 2013. Although these judgement documents are unstructured, we can simply use rules and regular expressions to extract the facts description, relevant law articles, and final charges of the case. This naturally provides us with a high-quality large-scale training dataset for our task. The experimental results show that our method significantly outperforms the baselie BOW-based Suport Vecotr Machine (SVM) method, and the automatically extracted relevant law articles can clearly improve the model with only facts as input. We also find that our model can effectively attend to the true relevant articles, which to some extent provides us with the legal support of our charge prediction.

% In this paper, we focus specifically on predicting the charges of criminal cases in China, \orange{which began to} officially publish the judgement documents on China Judgements Online\footnote{http://wenshu.court.gov.cn/} since 2013. Although these judgement documents are unstructured, we can use rules and regular expressions to extract the facts description, relevant law articles, and final charges of the case. This naturally provides us with a high-quality large-scale training dataset for our task.

This work offers an effective way to predict appropriate charges of a case based solely on the fact descriptions. 
Our main contributions are: 
(1) We propose a neural network model that can jointly utilize the facts and the automatically extracted relevant articles of a case for charge prediction along with the legal support in the form of relevant law articles.
(2) We propose that the public judgement documents provides natural high-quality large-scale training data for tasks like fact-based charge prediction and relevant article extraction. The experimental results on these data exhibit promising performance of our proposed method.
(3) By further evaluating our model on human labeled news data, we show that the model trained on judgement documents have reasonable generalization ability on the text written by people who are not legal practitioners. \todo{polish this paragraph, especially (2)}



